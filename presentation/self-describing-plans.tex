\documentclass{presentation}
\usepackage[nott]{inconsolata}

\title{Self-Describing Plans}
\author{Kyle Sunden \\ Blaise Thompson}

\institute{University of Wisconsin--Madison}
\date{2022-04-01}

\begin{document}
\maketitle

\section{Motivation}

\begin{frame}{Why does UW-Madison care?}
  \begin{itemize}
    \item Want ``first class'' user interfaces for our Bluesky systems
    \begin{itemize}
      \item typhos
      \item webclient
    \end{itemize}
    \item Mix of control systems
    \begin{itemize}
      \item zero percent EPICS
    \end{itemize}
    \item Way too small-scale to produce interfaces for each plan by hand
    \item Variety of plans
    \item Queueserver
  \end{itemize}
\end{frame}

\begin{frame}{Self-Describing Plans}
  Let's use annotations to make plans self-describing.
  \begin{itemize}
    \item{automate GUI generation}
    \item{validation}
    \item{serialize}
    \item{help queueserver cast correctly}
  \end{itemize}
\end{frame}

\section{Prior Art}

\begin{frame}[fragile]{Prior Art}
  Queueserver already supports plan descrptions through a home-built annotation system.
  \vfill

  \scriptsize
  \begin{minted}{JSON}
"name": "annotated_count",
"properties": {"is_generator": true},
"parameters": [
    {"name": "detectors",
      "kind": {"name": "POSITIONAL_OR_KEYWORD", "value": 1}},
    {"name": "num",
      "kind": {"name": "POSITIONAL_OR_KEYWORD", "value": 1},
      "annotation": {"type": "int"},
      "default": "1"},
    {"name": "per_shot",
      "kind": {"name": "KEYWORD_ONLY", "value": 3},
      "default": "None"},
    {"name": "md",
      "kind": {"name": "KEYWORD_ONLY", "value": 3},
      "default": "None"}
],
"module": "__main__"
  \end{minted}
\end{frame}

\begin{frame}{Prior Art}
  For those who need a review of parameter kinds enum (from inspect library)
  \begin{tabular}{ l | l }
    0 & \texttt{POSITIONAL\_ONLY} \\
    1 & \texttt{POSITIONAL\_OR\_KEYWORD} \\
    2 & \texttt{VAR\_POSITIONAL} \\
    3 & \texttt{KEYWORD\_ONLY} \\
    4 & \texttt{VAR\_KEYWORD} \\
  \end{tabular}
\end{frame}

\begin{frame}[fragile]{Prior Art}
  From queueserver documentation
  \vfill

  \scriptsize
  \begin{minted}{python}
from ophyd.sim import det1, det2, det3
# Assume that the detectors 'det1', 'det2', 'det3' are in the list
#   of allowed devices for the user submitting the plan.

from bluesky_queueserver import parameter_annotation_decorator

@parameter_annotation_decorator({
    "parameters": {
        "detectors": {
            "annotation": "typing.List[DevicesType1]",
            "devices": {"DevicesType1": ["det1", "det2", "det3"]}
        }
    }
})
def plan_demo1f(detectors, npts):
    # The parameter 'detector_names'
    #   is expected to receive a list of detector names.
    <code implementing the plan>
  \end{minted}
\end{frame}

\section{Proposal}

\begin{frame}{Proposal}
  Use typing to fully annotate plans. \\
  Simply annotate plans themselves as in PEP3107. \\

  \vfill

  for count:
  \begin{tabular}{ l | l }
    detectors & \texttt{typing.Sequence[bluesky.protocols.Readable]} \\
    num & \texttt{int} \\
    per\_shot & \texttt{typing.Callable} \\
    md & \texttt{typing.Dict[str, typing.Any]} \\
  \end{tabular}

  \vfill

  \begin{itemize}
    \item use standard library features, less code to maintain
    \item static type checking
    \item easy to inspect for serialization
    \item doesn't support queueserver concept ``limits'', but extensible
  \end{itemize}
\end{frame}

\begin{frame}[fragile]{Proposal}
  For those looking to automatically find all the plans in a namespace. \\
  Annotate return type.

  \vfill

  \scriptsize
  \begin{minted}{python}
from typing import Generator
def count(detectors, num=1, delay=None, *, per_shot=None, md=None)
    -> Generator[Msg, None, None]:
  \end{minted}

  \vfill

  \normalsize
  Would be easy to check return annotation for one of:

  \vfill

  \scriptsize
  \begin{minted}{python}
Sequence[Msg]
Generator[Msg, Any, Any]
  \end{minted}

  \vfill

  \normalsize
  There are probably other valid annotations that I'm forgetting. \\
  See bluesky/bluesky \#1491
\end{frame}

\begin{frame}[fragile]{Proposal}
  Put it all together...

  \vfill

  \scriptsize
  \begin{minted}{python}
def count(detectors: typing.Sequence[bluesky.protocols.Readable],
          num: int = 1,
          delay: typing.Union[NoneType,
                              typing.Iterable[float],
                              float]
                                = None,
          *,
          per_shot: typing.Callable = None,
          md: typing.Dict[str, typing.Any] = None)
    -> Generator[Msg, None, None]:
  \end{minted}

  \vfill
\end{frame}

\begin{frame}{Proposal}
  Problem: bluesky built-in plans make heavy use of \hl{variadic cycles}. \\
  Kyle and I cannot figure out how to hint these... \\
  Relevant PEPs:
  \begin{itemize}
    \item PEP3107: Function Annotations
    \item PEP593: Flexible function and variable annotations
    \item PEP612: Parameter Specification Variables
    \item PEP613: Explicit Type Aliases
    \item PEP646: Variadic Generics
    \item PEP484: Type Hints
  \end{itemize}
\end{frame}

\begin{frame}[fragile]{Proposal}
  Tuples for type checking, while remaining backwards compatible.
  \vfill

  \scriptsize
  \begin{minted}{python}
from bluesky import RunEngine
from bluesky import plans as bp
from ophyd.sim import det1, motor1, motor2
RE = RunEngine()
# still works
RE(bp.grid_scan([det1],
                motor1, -1.5, 1.5, 3,
                motor2, -0.1, 0.1, 5)
# now also valid
RE(bp.grid_scan([det1],
                (motor1, -1.5, 1.5, 3),
                (motor2, -0.1, 0.1, 5))
# passes mypy
RE(bp.grid_scan([det1],
                bp.GridScanAxis(motor1, -1.5, 1.5, 3),
                bp.GridScanAxis(motor2, -0.1, 0.1, 5))


  \end{minted}
\end{frame}

\begin{frame}{Proposal}
  for count:
  \begin{tabular}{ l | l }
    detectors & \texttt{typing.Sequence[bluesky.protocols.Readable]} \\
    num & \texttt{int} \\
    per\_shot & \texttt{typing.Callable} \\
    md & \texttt{typing.Dict[str, typing.Any]} \\
  \end{tabular}

  \vfill

  for grid\_scan:
  \begin{tabular}{ l | l }
    detectors & \texttt{typing.Sequence[bluesky.protocols.Readable]} \\
    args & \texttt{GridScanAxis} \\
    snake\_axes & \texttt{typing.Optional[bool]} \\
    per\_step & \texttt{typing.Callable} \\
    md & \texttt{typing.Dict[str, typing.Any]} \\
  \end{tabular}
\end{frame}

\begin{frame}{Proposal}
  Extend existing serialization system in Queueserver. \\
  Migrate serialize function into bluesky library.

  \vfill

  Cannot use existing schema standards, as far as we know... \\
  objects are too complex.

  \vfill
\end{frame}

\end{document}
